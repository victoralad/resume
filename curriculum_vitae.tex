%%%%%%%%%%%%%%%%%%%%%%%%%%%%%%%%%%%%%%%%%
% Medium Length Professional CV
% LaTeX Template
% Version 2.0 (8/5/13)
%
% This template has been downloaded from:
% http://www.LaTeXTemplates.com
%
% Original author:
% Trey Hunner (http://www.treyhunner.com/)
%
% Important note:
% This template requires the resume.cls file to be in the same directory as the
% .tex file. The resume.cls file provides the resume style used for structuring the
% document.
%
%%%%%%%%%%%%%%%%%%%%%%%%%%%%%%%%%%%%%%%%%

%----------------------------------------------------------------------------------------
%	PACKAGES AND OTHER DOCUMENT CONFIGURATIONS
%----------------------------------------------------------------------------------------

\documentclass{resume} % Use the custom resume.cls style

\usepackage[left=0.75in,top=0.6in,right=0.75in,bottom=0.6in]{geometry} % Document margins

\name{Victor Aladele} % Your name
%\address{1071A Terrell St. NW \\ Atlanta, GA 30318} % Your address

\address{(301)-379-5241} % Your phone number and email
\address{valad@gatech.edu}

\begin{document}

%----------------------------------------------------------------------------------------
%	EDUCATION SECTION
%----------------------------------------------------------------------------------------

\begin{rSection}{Education}

{\bf Georgia Institute of Technology} \hfill {\bf \em Atlanta GA} \\ 
PhD in Electrical Engineering \hfill {\em Aug 2016 - Present}\\
Anticipated Graduation \hfill {\em Aug 2022}\\
Research focus \hfill {\em Robotics and Machine Learning}\\
\\
{\bf New Jersey Institute of Technology} \hfill {\bf \em Newark NJ} \\ 
B.S. in Electrical Engineering \hfill {\em May 2016}\\
Minor in Applied Mathematics \\
Overall GPA: 3.76 (Magna Cum Laude)

\end{rSection}

%----------------------------------------------------------------------------------------
%	Publication SECTION
%----------------------------------------------------------------------------------------
\begin{rSection}{Publications}
\begin{itemize}
\item \textbf{V. Aladele} and S. Hutchinson, \textbf{Impedance-Based Collision Reaction Strategy via Internal Stress Loading in Cooperative Manipulation}, IEEE/RSJ International Conference on Intelligent Robots and Systems (IROS), 2021.

\item \textbf{V. Aladele} and S. Hutchinson, \textbf{Collision reaction through internal stress loading in cooperative manipulation}, IEEE/RSJ International Conference on Intelligent Robots and Systems (IROS), 2020.

\item Zafar, M., Mehmood, A., Khan, M., Zhang, S., Murtaza, M., \textbf{Aladele, V.}, Theodorou, E.A., Hutchinson, S. and Boots, B., 2018, November. \textbf{Semi-parametric Approaches to Learning in Model-Based Hierarchical Control of Complex Systems}. In International Symposium on Experimental Robotics (pp. 387-397). Springer, Cham.
\end{itemize}
\end{rSection}

%----------------------------------------------------------------------------------------
%	Internship Experience SECTION
%----------------------------------------------------------------------------------------
\begin{rSection}{Work Experience}
\begin{rSubsection}{Georgia Institute of Technology}{\bf \em Atlanta GA}{Graduate Research Assistant}{\em Aug 2016 - Present}
\item PhD Advisor: {\em Seth Hutchinson, PhD}
\item Dissertation Title: {\em Cooperative manipulation strategies for multi-robot and human-robot collaboration}
\end{rSubsection}

\begin{rSubsection}{Royal Institute of Technology (KTH)}{\bf \em Stockholm, Sweden}{Visiting PhD Student}{\em Aug 2021 - Jan 2022}
\item Host Advisor: {\em Danica Kragic Jensfelt, PhD}
\item Research focus: Compensating for model errors in cooperative manipulation: A Decentralized Approach.
\item Worked in Pybullet, using "Stable-Baselines" to train reinforcement learning agents to learn to compensate for model errors.
\end{rSubsection}

\begin{rSubsection}{Google (Brain/Research)}{\bf \em Remote / Mountain View CA}{Research Intern}{\em May 2021 - August 2021}
\item Worked on developing reinforcement learning solutions for high-speed robotics.
\item Developed and implemented curriculum learning algorithms to improve robot learning.
\item Implemented and trained different policy action spaces to improve performance of our robot.
\item Writing unittests.
\end{rSubsection}

\begin{rSubsection}{Blue River Technology (A John Deere Subsidiary)}{\bf \em Remote / Sunnyvale CA}{Software Engineering Intern}{\em May 2020 - Aug 2020}
\item Worked on a team to develop software for cutting-edge {\em John Deere} machinery
\item Worked in an agile-based development environment (Jira)
\item Unittesting with google testing framework
\item Worked with data serializing and deserializing frameworks such as: {\em Flatbuffers, Protocol buffers}
\item GPU programming, {\em CUDA}
\item Worked heavily with C++, including modern C++. 
\item Used Git with integrated testing (Jenkins) for version control
\end{rSubsection}

\begin{rSubsection}{Bosch (Advanced Corporate Research), BSH Home Appliances}{\bf \em Sunnyvale CA}{Robotics Software Intern}{\em May 2019 - Aug 2019}
\item Worked on implementing impedance control on a 6 DOF robotic arm for object insertion tasks.
\item Worked with different C++ libraries such as, Rigidbody Dynamics Library (RBDL). 
\item Wrote action-client ROS nodes for switching controllers. For example,  switching from a trajectory controller to an impedance controller.
\item Worked with C++, version control (Git), Python and ROS. 
\item Worked with both simulation and hardware.
\end{rSubsection}

\begin{rSubsection}{Massachusetts Institute of Technology}{\bf \em Cambridge MA}{Research Intern}{\em June 2015 - Aug 2015}
\item {\bf Advisors}: Daniela Rus {\em PhD},   Robert McCurdy, {\em PhD} \hfill {\bf CSAIL}
\item Designed and 3D printed gear pumps for hydraulically actuated robots.
\item Worked with Autodesk Inventor to design CAD models that were converted to STL files for
printing.
\end{rSubsection}



\end{rSection}

%----------------------------------------------------------------------------------------
%	TECHNICAL STRENGTHS SECTION
%----------------------------------------------------------------------------------------

\begin{rSection}{Technical Strengths}

\begin{tabular}{ @{} >{\bfseries}l @{\hspace{6ex}} l }
Computer Languages & C++, Python\\
Scripting Languages & HTML, XML, MATLAB \\
Tools & Robot Operating System (ROS), Pytorch, Git, Autodesk Inventor, \\ & Pybullet, OpenAI gym, Gazebo, Blender, Jupyter-notebook, CUDA
\end{tabular}

\end{rSection}

%----------------------------------------------------------------------------------------
%	RELEVANT COURSE SECTION
%----------------------------------------------------------------------------------------

\begin{rSection}{Relevant Courses}
Computer Vision  \hspace{20 pt} Machine Learning \hspace{26 pt} Stochastic Systems \hspace{10 pt}  Robot Intelligence and Planning
\\
Linear Systems \hspace{30 pt} Nonlinear Systems \hspace{20 pt} Optimal Control \hspace{22 pt} Interactive Robot Learning
\\
Advanced Programming Techniques ({\scriptsize CUDA, OpenMP, OpenGL, Sockets}) \hspace{10 pt} Mobile Manipulation


\end{rSection}

%------------------------------------------------

\begin{rSection}{Research Projects}

\begin{rSubsection}{Cooperative Mobile Manipulation}{\em August 2019 - Present}{Working both in simulation and on hardware}{}
\item Developing deep reinforcement learning schemes for multi-robot systems to cooperatively transport objects.
\item Implementing operational space control on KUKA IIWA7 arms that are mounted on mobile bases.
\item Implementing a vehicle-arm coordination scheme to enable the mobile base move in proper symphony with the arm.
\item Working with Rigidbody Dynamics Library (RBDL) and DRAKE in C++, version control (Git), Python and ROS.
\item Working with the following simulators: Gazebo, Drake and Pybullet (interfaced with OpenAI gym).
\end{rSubsection}

%------------------------------------------------
\end{rSection}

%----------------------------------------------------------------------------------------
%	PROJECTS SECTION
%----------------------------------------------------------------------------------------
\begin{rSection}{Class Projects}
%------------------------------------------------------------
\begin{rSubsection}{Robot Intelligence and Planning}{\em Fall 2020}{}{}
\item Implemented a version of DeepMind's AlphaZero chess AI. Used Deep Reinforcement Learning in conjunction with Monte-Carlo Tree Search to train a deep neural network to play the game of chess. Tools used include: Python, Pytorch, cuda.

\item Implemented deep reinforcement learning algorithms like: DQN, REINFORCE and A2C.

\item Implemented Rapidly-exploring Random Trees (RRT) to find a path between start and goal point on a 2D map. Algorithm implementation included steering dynamics with nonlinear optimization and obstacle detection. Code was written in Python.

\end{rSubsection}
%-----------------------------------------------------------
\begin{rSubsection}{Computer Vision}{\em Fall 2020}{}{}
\item Image classification using deep learning frameworks such as: CNNs, transfer learning with Alexnet and pytorch.
\item Feature Matching, using feature detectors (Harris detector) and feature descriptors (SIFT) in pytorch.
\end{rSubsection}
%-----------------------------------------------------------
\begin{rSubsection}{Advanced Programming Techniques}{\em Fall 2019}{}{}
\item Used OpenGL to design a bitmapped football field with multiple drones controlled by different MPI processes. The goal was to create a simulation of multiple drones display over a football field. 
\item Developed a distributed MPI program to guide simulated spaceships safely back to dock with the mothership (also simulated). Each spaceship was controlled by a different process, while the mothership was controlled by the master process. 
\item Designed a UDP server-client program.
\item Optimized code for solving 'Largest Product in a Grid' by creating an OpenMP multithreaded program.
\end{rSubsection}

%------------------------------------------------
\end{rSection}

%----------------------------------------------------------------------------------------
%	WORK EXPERIENCE SECTION
%----------------------------------------------------------------------------------------

\begin{rSection}{Teaching Positions}

\begin{rSubsection}{Graduate Teaching Assistant}{\em August 2016 - May 2018}{\em Georgia Tech}{Atlanta GA}
\item Signals and Systems, Junior year course (3 semesters)
\item Senior Design Project, Senior year course (2 semesters)
\end{rSubsection}

\end{rSection}


%----------------------------------------------------------------------------------------
%	HONORS SECTION
%----------------------------------------------------------------------------------------
\begin{rSection}{Honors, Awards and Societies}

\begin{rSubsection}{}{}{}{}
\item Tau Beta Pi Honors Society, Member \hfill {\em Aug 2014 - Present}
\item Institute of Electrical and Electronic Engineering, Member \hfill {\em Aug 2013 - Present}
\end{rSubsection}

\end{rSection}

\begin{rSection}{Extracurricular Activities}

\begin{rSubsection}{Volunteer High-school Curriculum Contributor}{\em March 2021}{}{}
\item Worked with Atlanta Public School teachers to develop a project-based learning (PBL) component of the Algebra II curriculum.
\end{rSubsection}

\begin{rSubsection}{Volunteer Application reviewer}{\em 2018 - present}{}{}
\item Annually review applications for the undergraduate summer research program at MIT.
\end{rSubsection}

\end{rSection}
%----------------------------------------------------------------------------------------
%	EXAMPLE SECTION
%----------------------------------------------------------------------------------------

%\begin{rSection}{Section Name}

%Section content\ldots

%\end{rSection}

%----------------------------------------------------------------------------------------

\end{document}
