%%%%%%%%%%%%%%%%%%%%%%%%%%%%%%%%%%%%%%%%%
% Medium Length Professional CV
% LaTeX Template
% Version 2.0 (8/5/13)
%
% This template has been downloaded from:
% http://www.LaTeXTemplates.com
%
% Original author:
% Trey Hunner (http://www.treyhunner.com/)
%
% Important note:
% This template requires the resume.cls file to be in the same directory as the
% .tex file. The resume.cls file provides the resume style used for structuring the
% document.
%
%%%%%%%%%%%%%%%%%%%%%%%%%%%%%%%%%%%%%%%%%

%----------------------------------------------------------------------------------------
%	PACKAGES AND OTHER DOCUMENT CONFIGURATIONS
%----------------------------------------------------------------------------------------

\documentclass{resume} % Use the custom resume.cls style

%\usepackage{hyperref}
\usepackage{xcolor}
\usepackage[colorlinks=true, citecolor=red, urlcolor=blue]{hyperref}
\usepackage[left=0.75in,top=0.6in,right=0.75in,bottom=0.6in]{geometry} % Document margins

\name{Victor Aladele} % Your name
%\address{1071A Terrell St. NW \\ Atlanta, GA 30318} % Your address

\address{voa2@njit.edu, +1(301)-379-5241, website: \href{https://victoralad.github.io}{victoralad.github.io}} % Your phone number and email
%\address{valad@gatech.edu}

\begin{document}

%----------------------------------------------------------------------------------------
%	EDUCATION SECTION
%----------------------------------------------------------------------------------------

\begin{rSection}{Education}

{\bf Georgia Institute of Technology} \hfill {\bf \em Atlanta GA} \\ 
PhD in Electrical Engineering \hfill {\em Aug 2016 - Present}\\
Graduated \hfill {\em Dec 2022}\\
Research focus \hfill {\em Robotics and Machine Learning}\\
\\
{\bf New Jersey Institute of Technology} \hfill {\bf \em Newark NJ} \\ 
B.S. in Electrical Engineering \hfill {\em May 2016}\\
Minor in Applied Mathematics \\
Overall GPA: 3.76 (Magna Cum Laude)

\end{rSection}

%----------------------------------------------------------------------------------------
%	Publication SECTION
%----------------------------------------------------------------------------------------
\begin{rSection}{Publications}
\begin{itemize}

\item \textbf{V. Aladele}, C. De Cos, D. Dimarogonas, S. Hutchinson, \textbf{An Adaptive Cooperative Manipulation Control Framework for Multi-Agent Disturbance Rejection}, IEEE Conference on Decision and Control (CDC), 2022 .

\item \textbf{V. Aladele} and S. Hutchinson, \textbf{Impedance-Based Collision Reaction Strategy via Internal Stress Loading in Cooperative Manipulation}, IEEE/RSJ International Conference on Intelligent Robots and Systems (IROS), 2021.

\item \textbf{V. Aladele} and S. Hutchinson, \textbf{Collision reaction through internal stress loading in cooperative manipulation}, IEEE/RSJ International Conference on Intelligent Robots and Systems (IROS), 2020.

\item M. Murtaza, \textbf{V. Aladele}, E. A. Theodorou, S. Hutchinson, and B. Boots, \textbf{Semi-parametric approaches to learning in model-based hierarchical control of complex
systems,} in Proceedings of the 2018 International Symposium on Experimental
Robotics (ISER), Springer Nature, vol. 11, 2020, p. 387.
\end{itemize}
\end{rSection}

%----------------------------------------------------------------------------------------
%	TECHNICAL SKILLS SECTION
%----------------------------------------------------------------------------------------

\begin{rSection}{Technical Skills}

\begin{tabular}{ @{} >{\bfseries}l @{\hspace{6ex}} l }
Programming Languages & C++, Python\\
Tools & Robot Operating System (ROS), Pytorch, Pybullet, Tensorflow \\ & OpenAI gym, MATLAB, Gazebo, Blender, CUDA, Autodesk Inventor
\end{tabular}

\end{rSection}
%----------------------------------------------------------------------------------------
%	Internship Experience SECTION
%----------------------------------------------------------------------------------------
\begin{rSection}{Work Experience}
\begin{rSubsection}{Fox Robotics}{\bf \em Austin TX}{Senior Software Engineer}{\em Aug 2022 - Present}
\item Developing motion planning and control algorithms for self-driving forklifts
\end{rSubsection}

\begin{rSubsection}{Georgia Institute of Technology}{\bf \em Atlanta GA}{Graduate Research Assistant}{\em Aug 2016 - Present}
\item PhD Advisor: {\em Seth Hutchinson, PhD}
\item Dissertation Title: {\em Cooperative manipulation strategies for multi-robot collaboration}
\end{rSubsection}

\begin{rSubsection}{Royal Institute of Technology (KTH)}{\bf \em Stockholm, Sweden}{Visiting PhD Student}{\em Aug 2021 - Jan 2022}
\item Host Advisor: {\em Danica Kragic Jensfelt, PhD}
\item Designed a novel application of residual reinforcement learning to cooperative manipulation. Tools used include: Pybullet, Stable-baselines, OpenAI gym
\end{rSubsection}

\begin{rSubsection}{Google (Brain/Research)}{\bf \em Remote / Mountain View CA}{Research Intern}{\em May 2021 - August 2021}
\item Worked on developing reinforcement learning solutions for high-speed robotics.
\item Developed and implemented curriculum learning algorithms to improve robot learning.
\end{rSubsection}

\begin{rSubsection}{Blue River Technology (A John Deere Subsidiary)}{\bf \em Remote / Sunnyvale CA}{Software Engineering Intern}{\em May 2020 - Aug 2020}
\item Worked on a team to develop software for cutting-edge {\em John Deere} machinery
\item Tools and frameworks used include: C++17, CUDA, Flatbuffers, Protocol buffers, Google Test, Jira.
\end{rSubsection}

\begin{rSubsection}{Bosch (Advanced Corporate Research), BSH Home Appliances}{\bf \em Sunnyvale CA}{Robotics Software Intern}{\em May 2019 - Aug 2019}
\item Worked on implementing impedance control on a 6 DOF robotic arm for object insertion tasks.
\item Tools used include: C++, Python, ROS, Rigidbody Dynamics Library (RBDL), Gazebo, Kinova arm.
\end{rSubsection}

\begin{rSubsection}{Massachusetts Institute of Technology}{\bf \em Cambridge MA}{Research Intern}{\em June 2015 - Aug 2015}
\item {\bf Advisors}: Daniela Rus {\em PhD},   Robert MacCurdy, {\em PhD} \hfill {\bf CSAIL}
\item Designed and 3D printed gear pumps for hydraulically actuated robots.
\item Designed CAD models in Autodesk Inventor.
\end{rSubsection}

\end{rSection}

%------------------------------------------------

\begin{rSection}{Research Projects}

\begin{rSubsection}{Cooperative Mobile Manipulation}{\em August 2019 - Present}{Working both in simulation and on hardware}{}
\item Developing deep reinforcement learning schemes for multi-robot collaboration.
\item Using TrajOpt for motion planning on a single-arm pick-and-place task.
\item Applying adaptive control for dual-arm disturbance rejection.
\item Implementing operational space control on KUKA IIWA7 arms that are mounted on mobile bases.
\item Applying Gaussian Processes as a semi-parametric control approach for a 7DOF manipulator.
\item Working in Gazebo, Pybullet, Matlab/Simulink and Drake.
\end{rSubsection}

%------------------------------------------------
\end{rSection}

%----------------------------------------------------------------------------------------
%	PROJECTS SECTION
%----------------------------------------------------------------------------------------
\begin{rSection}{Class Projects}
%------------------------------------------------------------
\begin{rSubsection}{Robot Intelligence and Planning}{\em Fall 2020}{}{}
\item Implemented a version of DeepMind's AlphaZero chess AI. Used Deep Reinforcement Learning in conjunction with Monte-Carlo Tree Search to train a deep neural network to play the game of chess. Tools used include: Python, Pytorch, cuda.

\item Implemented deep reinforcement learning algorithms like: DQN, REINFORCE and A2C.

\item Implemented Rapidly-exploring Random Trees (RRT) to navigate a 2D map. Algorithm implementation included steering dynamics with nonlinear optimization and obstacle detection.

\end{rSubsection}
%-----------------------------------------------------------
\begin{rSubsection}{Computer Vision}{\em Fall 2020}{}{}
\item Image classification using deep learning framework; transfer learning with CNNs like Alexnet.
\item Feature Matching, using feature detectors (Harris detector) and feature descriptors (SIFT) in pytorch.
\item Object detection with limited training data; applied transfer learning.
\end{rSubsection}
%-----------------------------------------------------------
\begin{rSubsection}{Advanced Programming Techniques}{\em Fall 2019}{}{}
\item Used OpenGL to simulate a bitmapped football field with multiple drones controlled by a distributed MPI program. The goal was to create a simulation of multiple drones display over a football field. 
\item Designed a UDP server-client program.
\end{rSubsection}

\begin{rSubsection}{Introduction to Robotics Research}{\em Fall 2017}{}{}
\item Target following and tracking using LIDAR, a camera and a PID controller on a turtlebot.
\item Obstacle avoidance using LIDAR and odometry. One of the tasks in this section was to drive
the robot to a target location using signs for direction along the way. This required using image
processing and classification techniques such as the ‘hough circles’ and ‘K Nearest Neighbors’
classifier.
\item Used Simultaneous Localization and Mapping (SLAM) with ROS ‘navstack’ for obstacle
avoidance.
\item Worked with ROS, python, OpenCV and Gazebo simulator on Linux platforms.
\end{rSubsection}

%------------------------------------------------
\end{rSection}

%----------------------------------------------------------------------------------------
%	RELEVANT COURSE SECTION
%----------------------------------------------------------------------------------------

\begin{rSection}{Relevant Courses}
Computer Vision  \hspace{20 pt} Machine Learning \hspace{26 pt} Stochastic Systems \hspace{10 pt}  Robot Intelligence and Planning
\\
Linear Systems \hspace{30 pt} Nonlinear Systems \hspace{20 pt} Optimal Control \hspace{22 pt} Interactive Robot Learning
\\
Advanced Programming Techniques ({\scriptsize CUDA, OpenMP, OpenGL, Sockets}) \hspace{10 pt} Mobile Manipulation


\end{rSection}


%----------------------------------------------------------------------------------------
%	WORK EXPERIENCE SECTION
%----------------------------------------------------------------------------------------

\begin{rSection}{Teaching Positions}

\begin{rSubsection}{Graduate Teaching Assistant}{\em August 2016 - May 2018}{\em Georgia Tech}{Atlanta GA}
\item Signals and Systems, Junior year course (3 semesters)
\item Senior Design Project, Senior year course (2 semesters)
\end{rSubsection}

\end{rSection}


%----------------------------------------------------------------------------------------
%	HONORS SECTION
%----------------------------------------------------------------------------------------
\begin{rSection}{Honors, Awards and Societies}

\begin{rSubsection}{}{}{}{}
\item Tau Beta Pi Honors Society, Member \hfill {\em August 2014 - Present}
\item Institute of Electrical and Electronic Engineering, Member \hfill {\em August 2013 - Present}
\item Selected to participate in an IDEO design-a-thon at the IEEE EMBS \hfill {\em Oct 2014, Seattle WA}\\ Special Topic Conference on Healthcare Innovation and Point-of-Care Technologies 
\end{rSubsection}

\end{rSection}

\begin{rSection}{Extracurricular Activities}

\begin{rSubsection}{Volunteer Application Reviewer}{\em 2018 - present}{}{}
\item Annually review applications for the undergraduate summer research program at MIT.
\end{rSubsection}

\begin{rSubsection}{Conference Publication Reviewer}{\em April 2020 - Present}{}{}
\item Reviewed papers for publication at the following conferences: IROS(2020, 2021, 2022), ICRA(2021).
\end{rSubsection}

\begin{rSubsection}{Conference Session Co-Chair}{\em September 2021}{}{}
\item Co-chaired the "Multi-Robot Systems I" session at the IROS 2021 conference.
\item Reviewed papers for publication at the following conferences: IROS(2020, 2021), ICRA(2021).
\end{rSubsection}

\begin{rSubsection}{Volunteer High-school Curriculum Contributor}{\em March 2021}{}{}
\item Worked with Atlanta Public School teachers to develop a project-based learning (PBL) component of the Algebra II curriculum.
\end{rSubsection}

\end{rSection}


%----------------------------------------------------------------------------------------
%	EXAMPLE SECTION
%----------------------------------------------------------------------------------------

%\begin{rSection}{Section Name}

%Section content\ldots

%\end{rSection}

%----------------------------------------------------------------------------------------

\end{document}
